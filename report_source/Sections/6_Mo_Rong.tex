% Update 5/12/2024
\vfil\penalty-200\vfilneg
\section{Thảo luận và Mở rộng}

\subsection{Đặc điểm của dữ liệu và các yêu cầu của phân tích hồi quy}
Phân tích hồi quy, đặc biệt là hồi quy tuyến tính và đa biến, yêu cầu thỏa mãn các giả định sau:

\begin{itemize}
    \item \textbf{Phân phối chuẩn của phần dư:}  
    Dữ liệu nhóm đã thực hiện kiểm định Shapiro-Wilk cho các biến như \textit{order\_price} và \textit{order\_total}. Kết quả ($p\text{-value} < 0.05$) cho thấy dữ liệu không tuân theo phân phối chuẩn ở từng nhóm mùa. Việc áp dụng hồi quy trong tình huống này có thể dẫn đến kết quả sai lệch.

    \item \textbf{Quan hệ tuyến tính giữa biến phụ thuộc và biến độc lập:}  
    Mối quan hệ giữa \textit{order\_total} và các yếu tố khác (như mùa, chi phí giao hàng) không đảm bảo tính tuyến tính rõ ràng. Dữ liệu có tính chất rời rạc và chịu ảnh hưởng của nhiều yếu tố phi tuyến tính như giảm giá và chi phí giao hàng.

    \item \textbf{Phương sai đồng nhất:}  
    Dữ liệu cho thấy sự khác biệt đáng kể trong độ biến thiên của \textit{order\_total} giữa các mùa (ví dụ, phí giao hàng mùa xuân cao hơn mùa đông), dẫn đến vi phạm giả định phương sai đồng nhất.

    \item \textbf{Không có sự tương quan:}  
    Các biến trong bộ dữ liệu có thể bị phụ thuộc, ví dụ, \textit{order\_price} bị ảnh hưởng bởi \textit{coupon\_discount} và \textit{delivery\_charges}, dẫn đến tự tương quan giữa các biến.
\end{itemize}

\noindent
\textbf{Kết luận:}  
Phân tích hồi quy tuyến tính và đa biến không thích hợp trong trường hợp này vì dữ liệu không đáp ứng các giả định cần thiết.

\subsection{Điểm mạnh của các kiểm định không tham số}

Các phương pháp như Kruskal-Wallis, Wilcoxon-Mann-Whitney, và Wilcoxon Signed Rank được thiết kế để hoạt động tốt với dữ liệu không chuẩn, có giá trị ngoại lai, hoặc không tuân theo phân phối tuyến tính.

\subsubsection{Kiểm định Kruskal-Wallis}
\begin{itemize}
    \item \textbf{Mục tiêu:}  
    So sánh sự khác biệt trung vị của \textit{order\_total} giữa bốn nhóm mùa (spring, summer, autumn, winter).
    \item \textbf{Kết quả:}  
    $p\text{-value} = 0.4266 > 0.05$, nghĩa là không có sự khác biệt có ý nghĩa thống kê giữa các mùa về giá trị đơn hàng.
    \item \textbf{Phù hợp vì:}  
    \begin{itemize}
        \item Không yêu cầu dữ liệu phải chuẩn.
        \item Phù hợp với dữ liệu phân loại như \textit{season}.
        \item Có khả năng xử lý dữ liệu với kích thước mẫu không đồng nhất giữa các nhóm.
    \end{itemize}
    % Giới hạn của phương pháp 1
    \textbf{Điểm yếu:}
\begin{itemize}
    \item Không cung cấp thông tin về sự khác biệt cụ thể giữa các nhóm, chỉ cho biết có sự khác biệt hay không.
    \item Cần phải thực hiện kiểm định sau (như Dunn’s Test) để xác định sự khác biệt giữa các nhóm cụ thể nếu $p\text{-value}$ nhỏ.
\end{itemize}

\end{itemize}

\subsubsection{Kiểm định Wilcoxon-Mann-Whitney}
\begin{itemize}
    \item \textbf{Mục tiêu:}  
    Kiểm tra sự khác biệt giữa \textit{order\_total} của hai mùa cụ thể (mùa xuân và mùa hè).

    \item \textbf{Kết quả:}  
    $p\text{-value} = 0.1701 > 0.05$, kết luận không có sự khác biệt đáng kể giữa hai mùa.
    \item \textbf{Phù hợp vì:}  
    \begin{itemize}
        \item Là phương pháp không tham số, không bị ảnh hưởng bởi phân phối bất thường hoặc ngoại lai.
        \item Được thiết kế để so sánh hai nhóm độc lập.
    \end{itemize}

\textbf{Điểm yếu:}
\begin{itemize}
    \item Chỉ so sánh giữa hai nhóm, không thể so sánh nhiều nhóm cùng lúc.
    \item Nhạy cảm với các giá trị ngoại lai, có thể làm ảnh hưởng đến kết quả.
\end{itemize}

\end{itemize}

\subsubsection{Kiểm định Wilcoxon Signed Rank}
\begin{itemize}
    \item \textbf{Mục tiêu:}  
    So sánh sự khác biệt giữa giá trị \textit{order\_price} (giá trị gốc) và \textit{order\_total} (giá trị thực tế sau khi áp dụng giảm giá và phí giao hàng).

    \item \textbf{Phù hợp vì:}  
    \begin{itemize}
        \item Là phương pháp dành riêng cho các cặp giá trị phụ thuộc.
        \item Không yêu cầu giả định phân phối chuẩn.
    \end{itemize}
    \item \textbf{Kết quả:}  
    $p\text{-value} = 0.042 < 0.05$, khẳng định có sự khác biệt có ý nghĩa thống kê giữa \textit{order\_price} và \textit{order\_total}. Sự khác biệt này phản ánh tác động của các yếu tố như giảm giá và chi phí giao hàng.
% Điểm yếu của kiểm định Wilconixin Signed Rank
\textbf{Điểm yếu:}
\begin{itemize}
    \item Chỉ áp dụng với dữ liệu có cặp (paired data), không thể áp dụng với dữ liệu độc lập.
    \item Độ chính xác của kiểm định có thể bị ảnh hưởng nếu có quá nhiều giá trị bằng nhau trong dữ liệu.
\end{itemize}
\end{itemize}

\subsection{Mối quan hệ giữa mục tiêu và phương pháp được chọn}

\begin{itemize}
    \item Phân tích hồi quy thường dùng để xây dựng mô hình dự đoán hoặc mô hình hóa mối quan hệ tuyến tính giữa biến phụ thuộc và biến độc lập.
    \item Trong khi đó, bài toán của nhóm tập trung vào việc kiểm tra:
    \begin{itemize}
        \item \textbf{Sự khác biệt giữa các nhóm (mùa):} Dùng Kruskal-Wallis và Wilcoxon-Mann-Whitney.
        \item \textbf{Sự khác biệt giữa các giá trị trong cùng một nhóm:} Dùng Wilcoxon Signed Rank.
    \end{itemize}
    \item Việc lựa chọn kiểm định không tham số phản ánh mục tiêu phân tích cụ thể mà không phụ thuộc vào các giả định phức tạp về phân phối dữ liệu.
\end{itemize}

\noindent
\textbf{Kết luận:}  
Phân tích hồi quy không được chọn vì:
\begin{itemize}
    \item Dữ liệu vi phạm các giả định quan trọng (phân phối chuẩn, phương sai đồng nhất).
    \item Mục tiêu của nhóm không phải dự đoán giá trị mà là kiểm tra sự khác biệt thống kê.
\end{itemize}
Các kiểm định không tham số được chọn phù hợp hơn, chính xác hơn, và cung cấp kết quả giải thích trực tiếp liên quan đến câu hỏi bài toán.

\subsection{Mở rộng: Dunn Test}
% Dàn ý:
% Mở rộng phương pháp Krustal-Wallis
% Nêu ra những điểm mạnh , điếm yếu của 3 phương pháp trên
Sau khi đưa ra những điểm mạnh hay giới hạn những phương pháp nhóm đã sử dụng ở trên. Nhóm quyết định sẽ mở rộng phương pháp \textbf{Krustal-Wallis} 
Như đề cập ở trên, phương pháp kiểm định KW chỉ đưa ra được "bề nổi" hay chỉ kiểm tra sự khác biệt tổng thể giữa các nhóm.

\noindent Nhóm sẽ áp dụng phương pháp hậu nghiệm (pos-hoc) để tìm ra cụ thể nhóm nào có sự khác biệt. Nhóm đề xuất sử dụng \textbf{Dunn Test}.

\begin{lstlisting}[language=R,caption=Thực hiện phân tích Pos Hoc trong R]
dunn_result <- dunn.test(merged_data$order_price, merged_data$season, method = "bonferroni", list = TRUE)
# Tạo bảng kết quả
table <- cbind.data.frame(
  Comparison = dunn_result$comparisons,  # Các cặp so sánh (cụ thể mùa)
  Z_value = dunn_result$Z,               # Giá trị Z
  P_adjusted = dunn_result$P.adjusted    # P-value đã điều chỉnh
)
# Sắp xếp bảng theo p-value đã điều chỉnh
table <- table[order(table$P_adjusted), ]
# Tạo bảng với gt và thêm tiêu đề
table %>%
  gt() %>%
  tab_header(
    title = md("#### Kết quả phân tích pos-hoc (Dunn's Test)"),
    subtitle = "So sánh sự khác nhau giữa các nhóm mùa bằng Bonerroni"
  )

\end{lstlisting}
\subsection{Kết quả}
% Table

\begin{table}[H]
\centering
\begin{tabular}{|l|r|r|} % Cột: l (trái), r (phải), r (phải)
\hline
\textbf{Các mùa} & \textbf{Z-value} & \textbf{P.adjusted} \\ \hline
autumn - winter     & 1.4580097        & 0.4345136           \\ \hline
autumn - spring     & 0.8823710        & 1.0000000           \\ \hline
autumn - summer     & 0.6008488        & 1.0000000           \\ \hline
spring - summer     & -0.2722941       & 1.0000000           \\ \hline
spring - winter     & 0.6104089        & 1.0000000           \\ \hline
summer - winter     & 0.8656661        & 1.0000000           \\ \hline
\end{tabular}
\caption{Post-hoc Analysis Results (Dunn's Test)}
\label{tab:dunn_results}
\end{table}
% End table

    \textbf{Kết luận:}
    Sau khi xuất ra kết quả Dunn Test, tổng quan các mùa đều có khác biệt gì lớn.
\section{Nguồn Dữ Liệu}
\begin{itemize}
% \href{http://www.overleaf.com}{Something Linky} 
    \item Dữ liệu mẫu: \href{https://www.kaggle.com/datasets/muhammadshahrayar/transactional-retail-dataset-of-electronics-store}{https://www.kaggle.com/datasets/muhammadshahrayar/transactional-retail-dataset-of-electronics-store}
    \item Link R của bài tập lớn: 
    \href{https://github.com/bnhtho/btl_xtsk_241/blob/main/2333017_Assigment.R}{https://github.com/bnhtho/btl\textunderscore{xtsk}\textunderscore{241}/blob/main/2333017\textunderscore{Assigment}.R}
    
\end{itemize}
\section{Tài liệu Tham Khảo}
% Tham khảo
\begin{thebibliography}{99}
\bibitem{anderson} 
Anderson, D. R., Sweeney, D. J., \& Williams, T. A. (2016). 
\textit{Statistics for Business and Economics} (11th ed.). Cengage Learning Vietnam Company Limited.

\bibitem{peterdaagard}
Peter Dalgaard.
\textit{Introductory Statistics with R} (8th ed)
\bibitem{nguyendinhhuy}
Nguyễn Đình Huy.
\textit{Giáo trình xác suất và thống kê(2023)}, Nxb. Đại học Quốc Gia, Thành phố Hồ Chí Minh.

\end{thebibliography}

