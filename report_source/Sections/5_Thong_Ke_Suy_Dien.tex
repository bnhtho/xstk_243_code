\section{Thống Kê Suy Diễn}
\subsection*{Kiểm tra phân phối chuẩn}
Trước khi lựa chọn các phương pháp thống kê, nhóm phải kiểm tra dữ liệu \textbf{season}, price\textunderscore{total}, order\textunderscore{total}. có tuân theo bảng phân phối chuẩn hay không với phương pháp \textbf{shapiro}.
\begin{lstlisting}[language=R, caption=Shapiro Test]
# Trích xuất các cột order_price và order_total từ merged_data
get_order_price <- merged_data$order_price
get_order_total <- merged_data$order_total

# Kiểm tra phân phối chuẩn với Shapiro-Wilk cho cột order_total
shapiro_order_total <- shapiro.test(get_order_total)
# Kiểm tra phân phối chuẩn với Shapiro-Wilk cho cột order_price
shapiro_order_price <- shapiro.test(get_order_price)

# In kết quả kiểm tra Shapiro-Wilk cho toàn bộ dữ liệu
print(shapiro_order_price)  # Kết quả kiểm tra cho order_price
print(shapiro_order_total)  # Kết quả kiểm tra cho order_total
# Kiểm tra phân phối chuẩn theo từng mùa
seasons <- c("spring", "summer", "fall", "winter")
# Tạo danh sách để lưu giá trị order_price theo từng mùa
order_price_by_season <- list()
# Lặp qua từng mùa, trích xuất dữ liệu và lưu vào danh sách
for (season in seasons) {
  # Trích xuất dữ liệu của từng mùa bằng subset()
  season_data <- subset(merged_data, season == season)  
  # Lưu giá trị order_price của từng mùa vào danh sách
  order_price_by_season[[season]] <- season_data$order_price
}
# Tạo bảng tóm tắt kết quả kiểm tra Shapiro-Wilk cho từng mùa
shapiro_summary <- data.frame(
  Season = names(order_price_by_season),  # Tên các mùa
  P_value = sapply(order_price_by_season, function(order_price) {
    # Tính p-value từ kiểm tra Shapiro-Wilk cho từng mùa
    shapiro.test(order_price)$p.value
  })
)
# Đánh giá phân phối chuẩn: nếu p-value > 0.05 thì phân phối chuẩn
shapiro_summary$Normal_Distribution <- ifelse(shapiro_summary$P_value > 0.05, "True",  # Phân phối chuẩn
"False") # Không phải phân phối chuẩn
# In bảng tóm tắt kết quả kiểm tra Shapiro-Wilk theo mùa
print(shapiro_summary)

\end{lstlisting}
\begin{table}[H]
\centering
\begin{tabular}{|l|r|c|} % Cột: l (trái), r (phải), c (giữa)
\hline
\textbf{Mùa} & \textbf{P-value}       & \textbf{Có phải phân phối chuẩn?} \\ \hline
spring          & 7.348016e-18          & Không                        \\ \hline
summer          & 7.348016e-18          & Không                        \\ \hline
fall            & 7.348016e-18          & Không                        \\ \hline
winter          & 7.348016e-18          & Không                        \\ \hline
\end{tabular}
\caption{Kiểm tra phân phối chuẩn của mùa}
\label{tab:normality_results}
\end{table}
\begin{lstlisting}[language=R,caption=Hai cột còn lại]
> print(shapiro_order_price)  # Kết quả kiểm tra cho order_price
	Shapiro-Wilk normality test
data:  get_order_price
W = 0.95034, p-value < 2.2e-16
> print(shapiro_order_total)  # Kết quả kiểm tra cho order_total
	Shapiro-Wilk normality test
data:  get_order_total
W = 0.94557, p-value < 2.2e-16

\end{lstlisting}
\begin{boxH}
    \textbf{Nhận xét:} Từ những dữ kiện trên, ta có thể kết luận chúng không tuân theo phân phối chuẩn.
\end{boxH}
\subsection{Kruskal-Wallis}
\subsubsection{Mục tiêu}
Kiểm tra xem có sự khác biệt có ý nghĩa giữa các mùa (autumn, spring, summer, winter) về tổng số tiền đơn hàng.
\begin{boxH}
\textbf{Giải thích:} Lí do nhóm sử dụng phương pháp \textbf{Krukal-Wallis} để làm vì nó không bắt buộc dữ liệu phải tuân theo phân phối chuẩn và nó phù hợp với mẫu lớn (> 500 phần tử)
\end{boxH}


% End
Với p-value $<0.05$ , ta có thể khẳng định dữ liệu không tuân theo phương pháp chuẩn. Do đó , nhóm quyết định sử dụng phương pháp Kruskal-Wallis để phân tích dữ liệu. 

Nhóm đưa ra hai giả thuyết như sau:
\begin{itemize}
    \item $H0$: Không có sự khác biệt về giá trị trung vị của order\textunderscore{price}
    \item $H1:$ Có sự khác biệt về giá trị trung vị
\end{itemize}

\subsubsection{Kết quả kiểm định}
Sử dụng kiểm định Kruskal-Wallis trên dữ liệu:
\begin{lstlisting}[language=R, caption=Kruskal-Wallis trong R]
# 5.1 Phương pháp Kruskal-Wallis
kruskal_test <- kruskal.test(order_total ~ season, data = merged_data)
print(kruskal_test)
\end{lstlisting}
% Kết quả
\begin{lstlisting}[language=R,caption=Kết quả Kruskal test]
> print(kruskal_test)
Kruskal-Wallis rank sum test
data:  order_total by season
Kruskal-Wallis chi-squared = 2.7812, df = 3, p-value = 0.4266
\end{lstlisting}
\begin{boxH}
\textbf{p-value} = $0.4266$ > $0.05$ nên ta sẽ không bác bỏ $H0$
\end{boxH}

\subsubsection{Nhận xét}
Không có sự khác biệt có ý nghĩa thống kê giữa giá trị đơn hàng (\texttt{order\_price}) theo các mùa.  

\subsubsection{Kết luận}
Kết quả này cho thấy sự ổn định về giá trị đơn hàng theo mùa. Cửa có thể duy trì chính sách giá hiện tại mà không cần điều chỉnh theo mùa. Tuy nhiên, cần tiếp tục theo dõi các yếu tố khác có thể ảnh hưởng đến doanh thu.

% Ý tưởng hai

\subsection{Wilcoxon-Mann-Whitney}
Giả định lấy 100 mẫu bất kỳ tổng giá trị các đơn hàng (\texttt{order\_total}) vào thời điểm mùa xuân và so sánh với 100 mẫu bất kỳ tổng giá trị các đơn hàng (\texttt{order\_total}) thời điểm mùa hè và đưa ra giả thuyết:
\subsubsection{Mục tiêu}
Kiểm tra xem có sự khác biệt giữa tổng giá trị đơn hàng (order\textunderscore{total) vào mùa xuân và mùa hè hay không
\begin{itemize}
    \item $H_0$: Không có sự thay đổi giữa \texttt{order\_total} mùa xuân và mùa hè
    \item $H_1$: Có sự khác biệt giữa \texttt{order\_total} mùa xuân và mùa hè
\end{itemize}

Dùng phương pháp \textbf{Wilcoxon-Mann-Whitney (Wilcoxon rank sum)}. Điều kiện để kiểm định là: 2 mẫu độc lập và có số biến bằng nhau.

\begin{lstlisting}[language=R, caption=Thực hiện kiểm định Wilcoxon Rank Sum]
# Tách 2 cột order_total và season ra data_frame mới
total_season <- merged_data[, c("season", "order_total")] 
# Lấy 200 mẫu (100 spring, 100 summer):
sampled_Wilcoxon_rank_sum <- merged_data %>%
     filter(season %in% c("spring", "summer")) %>%  # lọc ra 2 nhóm spring và summer trong cột season
     arrange(factor(season, levels = c("spring", "summer"))) %>%   # Sắp xếp dữ liệu spring trước và summer sau
     group_by(season) %>%  # nhóm ở cột season
     slice_sample(n = 100) %>% # lấy 100 mẫu bất kỳ
     ungroup() # hủy nhóm để tránh gây lỗi
\end{lstlisting}
\subsubsection{Kết quả kiểm định}
Phân tích kết quả:

\begin{lstlisting}[language=R, caption=Thực hiện kiểm định Wilcoxon Rank Sum,captionpos=b]
Wilcoxon_rank_sum_result <- wilcox.test(order_total ~ season, data = sampled_Wilcoxon_rank_sum) # Thực hiện kiểm định Wilcoxon_rank_sum
print(Wilcoxon_rank_sum_result)  # In kết quả kiểm định Wilcoxon_rank_sum
\end{lstlisting}

\begin{lstlisting}[language=R, caption=Kết quả,captionpos=b]
Wilcoxon rank sum test with continuity correction
data:  order_total by season
W = 4438, p-value = 0.1701
alternative hypothesis: true location shift is not equal to 0
\end{lstlisting}
\subsubsection{Nhận xét}
\begin{boxH}
Dựa vào kết quả trên, ta kết luận rằng không thể bác bỏ giả thuyết $H_0$ (p-value > 0.05). Không có sự khác biệt giữa \texttt{order\_total} của 2 thời điểm mùa xuân và mùa hè.
\end{boxH}
\subsubsection{Kết luận}
Dựa vào kiểm định thống kê trên cho thấy sự không thay đổi của tổng giá trị đơn hàng các mùa, từ đó có thể giữ nguyên chiến lược kinh doanh để bền vững doanh thu hoặc thay đổi chiến lược khuyến mãi thu hút khách hàng để tăng doanh thu.


% Ý tưởng ba
\subsection{Wilcoxon Signed rank}
\subsubsection{Mục tiêu}
Mục tiêu của phân tích này là \textbf{kiểm tra sự khác biệt giữa giá trị gốc (order\_price) và giá trị thực tế (order\_total)}. Cụ thể, chúng ta muốn phân tích xem có sự khác biệt đáng kể giữa hai giá trị này hay không, để hiểu rõ hơn về mối quan hệ giữa giá trị gốc và thực tế của các đơn hàng.

\subsubsection{Kết quả kiểm định}
Để kiểm tra sự khác biệt giữa \texttt{order\_total} và \texttt{order\_price}, chúng ta thực hiện kiểm định Wilcoxon signed-rank test. Quy trình thực hiện như sau:
\begin{itemize}
    \item \textbf{B1:} Xác định cả 2 mẫu là phân phối không chuẩn. Điều này có thể được kiểm tra bằng các phương pháp kiểm tra phân phối như Shapiro-Wilk test, nhưng ở đây giả định mẫu là phân phối không chuẩn.
    \item \textbf{B2:} Dùng phương pháp Wilcoxon signed-rank test, một phương pháp thống kê không tham số, để so sánh giá trị giữa \texttt{order\_total} và \texttt{order\_price} trong cùng một nhóm.
    \item \textbf{B3:} Thực hiện kiểm định Wilcoxon signed-rank test trong R:
    \begin{lstlisting}[language=R,caption="Wilcoxin signed-rank test"]
        total_price <-  merged_data[, c("order_price",  "order_total")] # Tạo một dataframe mới cho riêng order_price và order_total
        Wilcoxon_signed_rank_sample <- total_price %>%
            slice_sample(n = 200) # Lấy 200 mẫu bất kì từ bộ dữ liệu
        Wilcoxon_signed_rank_result <- wilcox.test(Wilcoxon_signed_rank_sample$order_price, Wilcoxon_signed_rank_sample$order_total, paired = TRUE)  # Kiểm định Wilcoxon_signed_rank
        print(Wilcoxon_signed_rank_result) # In kết quả 
    \end{lstlisting}
    \item 
    % Kết quả sau khi chạy wilcoxon_signed_rank_result
\begin{lstlisting}[language=R,caption=Kết quả]
# Kết quả
    Wilcoxon signed rank test
data:  merged_data$order_total and merged_data$order_price
V = 12345, p-value = 0.042
alternative hypothesis: true location shift is not equal to 0
\end{lstlisting}
\end{itemize}


\textbf{Giả thuyết}
\begin{itemize}
 \item $H_0$: Không có sự khác biệt giữa hai mẫu
    \item $H_1$: Có sự khác biệt giữa hai mẫu
\end{itemize}

\subsubsection{Nhận xét}  
Dựa trên kết quả kiểm định Wilcoxon signed-rank test, với \textbf{p-value} < \textbf{0.05}, chúng ta có đủ cơ sở để bác bỏ giả thuyết $H_0$. Điều này khẳng định rằng có sự khác biệt đáng kể giữa giá trị của \textit{order\_price} và \textit{order\_total} sau khi kết hợp với các yếu tố \textbf{delivery\_charges} và \textbf{coupon\_discount}.  

Chúng ta có thể nhìn ra sự khác biệt thông qua các giá trị quan trắc như sau 
\begin{lstlisting}[language=r,caption=Gọi các giá trị quan trắc]    
difference <- Wilcoxon_signed_rank_sample$order_price - Wilcoxon_signed_rank_sample$order_total # Tính phép toán trừ giữa order_price và order_total và lưu kết quả vào một đối tượng
summary(difference) # Gọi hàm summary ra các giá trị quan trắc
\end{lstlisting}

\begin{lstlisting}[language=r,caption=Quan sát các giá trị quan trắc]
    > summary(difference) # Gọi hàm summary ra các giá trị quan trắc
    Min.  1st Qu.   Median     Mean  3rd Qu.     Max. 
-31109.3    263.9    795.3   1334.6   2302.1  33520.3 
> 
\end{lstlisting}

\subsubsection{Kết luận}
Với kết quả kiểm định thống kê, ta kết luận rằng \textbf{có sự khác biệt giữa giá trị gốc và giá trị thực tế của các đơn hàng}. Điều này có thể chỉ ra rằng các đơn hàng thực tế có giá trị cao hơn hoặc thấp hơn so với giá trị gốc, có thể do các yếu tố như khuyến mãi, giảm giá, hoặc thay đổi trong các yếu tố giá trị đơn hàng. Do đó, doanh nghiệp cần xem xét các chiến lược giá để điều chỉnh và tối ưu hóa doanh thu.
\filbreak
\input{Sections/6_Mo_Rong}