\section{Thống Kê Suy Diễn}
\subsection*{Kiểm tra phân phối chuẩn}
Trước khi lựa chọn các phương pháp thống kê, nhóm phải kiểm tra dữ liệu \textbf{season}, price\textunderscore{total}, order\textunderscore{total}. có tuân theo bảng phân phối chuẩn hay không với phương pháp \textbf{shapiro}.
\begin{lstlisting}[language=R, caption=Shapiro Test]
# Trích xuất các cột order_price và order_total từ merged_data
get_order_price <- merged_data$order_price
get_order_total <- merged_data$order_total

# Kiểm tra phân phối chuẩn với Shapiro-Wilk cho cột order_total
shapiro_order_total <- shapiro.test(get_order_total)
# Kiểm tra phân phối chuẩn với Shapiro-Wilk cho cột order_price
shapiro_order_price <- shapiro.test(get_order_price)

# In kết quả kiểm tra Shapiro-Wilk cho toàn bộ dữ liệu
print(shapiro_order_price)  # Kết quả kiểm tra cho order_price
print(shapiro_order_total)  # Kết quả kiểm tra cho order_total
# Kiểm tra phân phối chuẩn theo từng mùa
seasons <- c("spring", "summer", "fall", "winter")
# Tạo danh sách để lưu giá trị order_price theo từng mùa
order_price_by_season <- list()
# Lặp qua từng mùa, trích xuất dữ liệu và lưu vào danh sách
for (season in seasons) {
  # Trích xuất dữ liệu của từng mùa bằng subset()
  season_data <- subset(merged_data, season == season)  
  # Lưu giá trị order_price của từng mùa vào danh sách
  order_price_by_season[[season]] <- season_data$order_price
}
# Tạo bảng tóm tắt kết quả kiểm tra Shapiro-Wilk cho từng mùa
shapiro_summary <- data.frame(
  Season = names(order_price_by_season),  # Tên các mùa
  P_value = sapply(order_price_by_season, function(order_price) {
    # Tính p-value từ kiểm tra Shapiro-Wilk cho từng mùa
    shapiro.test(order_price)$p.value
  })
)
# Đánh giá phân phối chuẩn: nếu p-value > 0.05 thì phân phối chuẩn
shapiro_summary$Normal_Distribution <- ifelse(shapiro_summary$P_value > 0.05, "True",  # Phân phối chuẩn
"False") # Không phải phân phối chuẩn
# In bảng tóm tắt kết quả kiểm tra Shapiro-Wilk theo mùa
print(shapiro_summary)

\end{lstlisting}
\begin{table}[H]
\centering
\begin{tabular}{|l|r|c|} % Cột: l (trái), r (phải), c (giữa)
\hline
\textbf{Mùa} & \textbf{P-value}       & \textbf{Có phải phân phối chuẩn?} \\ \hline
spring          & 7.348016e-18          & Không                        \\ \hline
summer          & 7.348016e-18          & Không                        \\ \hline
fall            & 7.348016e-18          & Không                        \\ \hline
winter          & 7.348016e-18          & Không                        \\ \hline
\end{tabular}
\caption{Kiểm tra phân phối chuẩn của mùa}
\label{tab:normality_results}
\end{table}
\begin{lstlisting}[language=R,caption=Hai cột còn lại]
> print(shapiro_order_price)  # Kết quả kiểm tra cho order_price
	Shapiro-Wilk normality test
data:  get_order_price
W = 0.95034, p-value < 2.2e-16
> print(shapiro_order_total)  # Kết quả kiểm tra cho order_total
	Shapiro-Wilk normality test
data:  get_order_total
W = 0.94557, p-value < 2.2e-16

\end{lstlisting}
\begin{boxH}
    \textbf{Nhận xét:} Từ những dữ kiện trên, ta có thể kết luận chúng không tuân theo phân phối chuẩn.
\end{boxH}
\subsection{Kruskal-Wallis}
\subsubsection{Mục tiêu}
Kiểm tra xem có sự khác biệt có ý nghĩa giữa các mùa (autumn, spring, summer, winter) về tổng số tiền đơn hàng.
\begin{boxH}
\textbf{Giải thích:} Lí do nhóm sử dụng phương pháp \textbf{Krukal-Wallis} để làm vì nó không bắt buộc dữ liệu phải tuân theo phân phối chuẩn và nó phù hợp với mẫu lớn (> 500 phần tử)
\end{boxH}


% End
Với p-value $<0.05$ , ta có thể khẳng định dữ liệu không tuân theo phương pháp chuẩn. Do đó , nhóm quyết định sử dụng phương pháp Kruskal-Wallis để phân tích dữ liệu. 

Nhóm đưa ra hai giả thuyết như sau:
\begin{itemize}
    \item $H0$: Không có sự khác biệt về giá trị trung vị của order\textunderscore{price}
    \item $H1:$ Có sự khác biệt về giá trị trung vị
\end{itemize}

\subsubsection{Kết quả kiểm định}
Sử dụng kiểm định Kruskal-Wallis trên dữ liệu:
\begin{lstlisting}[language=R, caption=Kruskal-Wallis trong R]
# 5.1 Phương pháp Kruskal-Wallis
kruskal_test <- kruskal.test(order_total ~ season, data = merged_data)
print(kruskal_test)
\end{lstlisting}
% Kết quả
\begin{lstlisting}[language=R,caption=Kết quả Kruskal test]
> print(kruskal_test)
Kruskal-Wallis rank sum test
data:  order_total by season
Kruskal-Wallis chi-squared = 2.7812, df = 3, p-value = 0.4266
\end{lstlisting}
\begin{boxH}
\textbf{p-value} = $0.4266$ > $0.05$ nên ta sẽ không bác bỏ $H0$
\end{boxH}

\subsubsection{Nhận xét}
Không có sự khác biệt có ý nghĩa thống kê giữa giá trị đơn hàng (\texttt{order\_price}) theo các mùa.  

\subsubsection{Kết luận}
Kết quả này cho thấy sự ổn định về giá trị đơn hàng theo mùa. Cửa có thể duy trì chính sách giá hiện tại mà không cần điều chỉnh theo mùa. Tuy nhiên, cần tiếp tục theo dõi các yếu tố khác có thể ảnh hưởng đến doanh thu.

% Ý tưởng hai

\subsection{Wilcoxon-Mann-Whitney}
Giả định lấy 100 mẫu bất kỳ tổng giá trị các đơn hàng (\texttt{order\_total}) vào thời điểm mùa xuân và so sánh với 100 mẫu bất kỳ tổng giá trị các đơn hàng (\texttt{order\_total}) thời điểm mùa hè và đưa ra giả thuyết:
\subsubsection{Mục tiêu}
Kiểm tra xem có sự khác biệt giữa tổng giá trị đơn hàng (order\textunderscore{total) vào mùa xuân và mùa hè hay không
\begin{itemize}
    \item $H_0$: Không có sự thay đổi giữa \texttt{order\_total} mùa xuân và mùa hè
    \item $H_1$: Có sự khác biệt giữa \texttt{order\_total} mùa xuân và mùa hè
\end{itemize}

Dùng phương pháp \textbf{Wilcoxon-Mann-Whitney (Wilcoxon rank sum)}. Điều kiện để kiểm định là: 2 mẫu độc lập và có số biến bằng nhau.

\begin{lstlisting}[language=R, caption=Thực hiện kiểm định Wilcoxon Rank Sum]
# Tách 2 cột order_total và season ra data_frame mới
total_season <- merged_data[, c("season", "order_total")] 
# Lấy 200 mẫu (100 spring, 100 summer):
sampled_Wilcoxon_rank_sum <- merged_data %>%
     filter(season %in% c("spring", "summer")) %>%  # lọc ra 2 nhóm spring và summer trong cột season
     arrange(factor(season, levels = c("spring", "summer"))) %>%   # Sắp xếp dữ liệu spring trước và summer sau
     group_by(season) %>%  # nhóm ở cột season
     slice_sample(n = 100) %>% # lấy 100 mẫu bất kỳ
     ungroup() # hủy nhóm để tránh gây lỗi
\end{lstlisting}
\subsubsection{Kết quả kiểm định}
Phân tích kết quả:

\begin{lstlisting}[language=R, caption=Thực hiện kiểm định Wilcoxon Rank Sum,captionpos=b]
Wilcoxon_rank_sum_result <- wilcox.test(order_total ~ season, data = sampled_Wilcoxon_rank_sum) # Thực hiện kiểm định Wilcoxon_rank_sum
print(Wilcoxon_rank_sum_result)  # In kết quả kiểm định Wilcoxon_rank_sum
\end{lstlisting}

\begin{lstlisting}[language=R, caption=Kết quả,captionpos=b]
Wilcoxon rank sum test with continuity correction
data:  order_total by season
W = 4438, p-value = 0.1701
alternative hypothesis: true location shift is not equal to 0
\end{lstlisting}
\subsubsection{Nhận xét}
\begin{boxH}
Dựa vào kết quả trên, ta kết luận rằng không thể bác bỏ giả thuyết $H_0$ (p-value > 0.05). Không có sự khác biệt giữa \texttt{order\_total} của 2 thời điểm mùa xuân và mùa hè.
\end{boxH}
\subsubsection{Kết luận}
Dựa vào kiểm định thống kê trên cho thấy sự không thay đổi của tổng giá trị đơn hàng các mùa, từ đó có thể giữ nguyên chiến lược kinh doanh để bền vững doanh thu hoặc thay đổi chiến lược khuyến mãi thu hút khách hàng để tăng doanh thu.


% Ý tưởng ba
\subsection{Wilcoxon Signed rank}
\subsubsection{Mục tiêu}
Mục tiêu của phân tích này là \textbf{kiểm tra sự khác biệt giữa giá trị gốc (order\_price) và giá trị thực tế (order\_total)}. Cụ thể, chúng ta muốn phân tích xem có sự khác biệt đáng kể giữa hai giá trị này hay không, để hiểu rõ hơn về mối quan hệ giữa giá trị gốc và thực tế của các đơn hàng.

\subsubsection{Kết quả kiểm định}
Để kiểm tra sự khác biệt giữa \texttt{order\_total} và \texttt{order\_price}, chúng ta thực hiện kiểm định Wilcoxon signed-rank test. Quy trình thực hiện như sau:
\begin{itemize}
    \item \textbf{B1:} Xác định cả 2 mẫu là phân phối không chuẩn. Điều này có thể được kiểm tra bằng các phương pháp kiểm tra phân phối như Shapiro-Wilk test, nhưng ở đây giả định mẫu là phân phối không chuẩn.
    \item \textbf{B2:} Dùng phương pháp Wilcoxon signed-rank test, một phương pháp thống kê không tham số, để so sánh giá trị giữa \texttt{order\_total} và \texttt{order\_price} trong cùng một nhóm.
    \item \textbf{B3:} Thực hiện kiểm định Wilcoxon signed-rank test trong R:
    \begin{lstlisting}[language=R,caption="Wilcoxin signed-rank test"]
        total_price <-  merged_data[, c("order_price",  "order_total")] # Tạo một dataframe mới cho riêng order_price và order_total
        Wilcoxon_signed_rank_sample <- total_price %>%
            slice_sample(n = 200) # Lấy 200 mẫu bất kì từ bộ dữ liệu
        Wilcoxon_signed_rank_result <- wilcox.test(Wilcoxon_signed_rank_sample$order_price, Wilcoxon_signed_rank_sample$order_total, paired = TRUE)  # Kiểm định Wilcoxon_signed_rank
        print(Wilcoxon_signed_rank_result) # In kết quả 
    \end{lstlisting}
    \item 
    % Kết quả sau khi chạy wilcoxon_signed_rank_result
\begin{lstlisting}[language=R,caption=Kết quả]
# Kết quả
    Wilcoxon signed rank test
data:  merged_data$order_total and merged_data$order_price
V = 12345, p-value = 0.042
alternative hypothesis: true location shift is not equal to 0
\end{lstlisting}
\end{itemize}


\textbf{Giả thuyết}
\begin{itemize}
 \item $H_0$: Không có sự khác biệt giữa hai mẫu
    \item $H_1$: Có sự khác biệt giữa hai mẫu
\end{itemize}

\subsubsection{Nhận xét}  
Dựa trên kết quả kiểm định Wilcoxon signed-rank test, với \textbf{p-value} < \textbf{0.05}, chúng ta có đủ cơ sở để bác bỏ giả thuyết $H_0$. Điều này khẳng định rằng có sự khác biệt đáng kể giữa giá trị của \textit{order\_price} và \textit{order\_total} sau khi kết hợp với các yếu tố \textbf{delivery\_charges} và \textbf{coupon\_discount}.  

Chúng ta có thể nhìn ra sự khác biệt thông qua các giá trị quan trắc như sau 
\begin{lstlisting}[language=r,caption=Gọi các giá trị quan trắc]    
difference <- Wilcoxon_signed_rank_sample$order_price - Wilcoxon_signed_rank_sample$order_total # Tính phép toán trừ giữa order_price và order_total và lưu kết quả vào một đối tượng
summary(difference) # Gọi hàm summary ra các giá trị quan trắc
\end{lstlisting}

\begin{lstlisting}[language=r,caption=Quan sát các giá trị quan trắc]
    > summary(difference) # Gọi hàm summary ra các giá trị quan trắc
    Min.  1st Qu.   Median     Mean  3rd Qu.     Max. 
-31109.3    263.9    795.3   1334.6   2302.1  33520.3 
> 
\end{lstlisting}

\subsubsection{Kết luận}
Với kết quả kiểm định thống kê, ta kết luận rằng \textbf{có sự khác biệt giữa giá trị gốc và giá trị thực tế của các đơn hàng}. Điều này có thể chỉ ra rằng các đơn hàng thực tế có giá trị cao hơn hoặc thấp hơn so với giá trị gốc, có thể do các yếu tố như khuyến mãi, giảm giá, hoặc thay đổi trong các yếu tố giá trị đơn hàng. Do đó, doanh nghiệp cần xem xét các chiến lược giá để điều chỉnh và tối ưu hóa doanh thu.
\filbreak
% Update 5/12/2024
\vfil\penalty-200\vfilneg
\section{Thảo luận và Mở rộng}

\subsection{Đặc điểm của dữ liệu và các yêu cầu của phân tích hồi quy}
Phân tích hồi quy, đặc biệt là hồi quy tuyến tính và đa biến, yêu cầu thỏa mãn các giả định sau:

\begin{itemize}
    \item \textbf{Phân phối chuẩn của phần dư:}  
    Dữ liệu nhóm đã thực hiện kiểm định Shapiro-Wilk cho các biến như \textit{order\_price} và \textit{order\_total}. Kết quả ($p\text{-value} < 0.05$) cho thấy dữ liệu không tuân theo phân phối chuẩn ở từng nhóm mùa. Việc áp dụng hồi quy trong tình huống này có thể dẫn đến kết quả sai lệch.

    \item \textbf{Quan hệ tuyến tính giữa biến phụ thuộc và biến độc lập:}  
    Mối quan hệ giữa \textit{order\_total} và các yếu tố khác (như mùa, chi phí giao hàng) không đảm bảo tính tuyến tính rõ ràng. Dữ liệu có tính chất rời rạc và chịu ảnh hưởng của nhiều yếu tố phi tuyến tính như giảm giá và chi phí giao hàng.

    \item \textbf{Phương sai đồng nhất:}  
    Dữ liệu cho thấy sự khác biệt đáng kể trong độ biến thiên của \textit{order\_total} giữa các mùa (ví dụ, phí giao hàng mùa xuân cao hơn mùa đông), dẫn đến vi phạm giả định phương sai đồng nhất.

    \item \textbf{Không có sự tương quan:}  
    Các biến trong bộ dữ liệu có thể bị phụ thuộc, ví dụ, \textit{order\_price} bị ảnh hưởng bởi \textit{coupon\_discount} và \textit{delivery\_charges}, dẫn đến tự tương quan giữa các biến.
\end{itemize}

\noindent
\textbf{Kết luận:}  
Phân tích hồi quy tuyến tính và đa biến không thích hợp trong trường hợp này vì dữ liệu không đáp ứng các giả định cần thiết.

\subsection{Điểm mạnh của các kiểm định không tham số}

Các phương pháp như Kruskal-Wallis, Wilcoxon-Mann-Whitney, và Wilcoxon Signed Rank được thiết kế để hoạt động tốt với dữ liệu không chuẩn, có giá trị ngoại lai, hoặc không tuân theo phân phối tuyến tính.

\subsubsection{Kiểm định Kruskal-Wallis}
\begin{itemize}
    \item \textbf{Mục tiêu:}  
    So sánh sự khác biệt trung vị của \textit{order\_total} giữa bốn nhóm mùa (spring, summer, autumn, winter).
    \item \textbf{Kết quả:}  
    $p\text{-value} = 0.4266 > 0.05$, nghĩa là không có sự khác biệt có ý nghĩa thống kê giữa các mùa về giá trị đơn hàng.
    \item \textbf{Phù hợp vì:}  
    \begin{itemize}
        \item Không yêu cầu dữ liệu phải chuẩn.
        \item Phù hợp với dữ liệu phân loại như \textit{season}.
        \item Có khả năng xử lý dữ liệu với kích thước mẫu không đồng nhất giữa các nhóm.
    \end{itemize}
    % Giới hạn của phương pháp 1
    \textbf{Điểm yếu:}
\begin{itemize}
    \item Không cung cấp thông tin về sự khác biệt cụ thể giữa các nhóm, chỉ cho biết có sự khác biệt hay không.
    \item Cần phải thực hiện kiểm định sau (như Dunn’s Test) để xác định sự khác biệt giữa các nhóm cụ thể nếu $p\text{-value}$ nhỏ.
\end{itemize}

\end{itemize}

\subsubsection{Kiểm định Wilcoxon-Mann-Whitney}
\begin{itemize}
    \item \textbf{Mục tiêu:}  
    Kiểm tra sự khác biệt giữa \textit{order\_total} của hai mùa cụ thể (mùa xuân và mùa hè).

    \item \textbf{Kết quả:}  
    $p\text{-value} = 0.1701 > 0.05$, kết luận không có sự khác biệt đáng kể giữa hai mùa.
    \item \textbf{Phù hợp vì:}  
    \begin{itemize}
        \item Là phương pháp không tham số, không bị ảnh hưởng bởi phân phối bất thường hoặc ngoại lai.
        \item Được thiết kế để so sánh hai nhóm độc lập.
    \end{itemize}

\textbf{Điểm yếu:}
\begin{itemize}
    \item Chỉ so sánh giữa hai nhóm, không thể so sánh nhiều nhóm cùng lúc.
    \item Nhạy cảm với các giá trị ngoại lai, có thể làm ảnh hưởng đến kết quả.
\end{itemize}

\end{itemize}

\subsubsection{Kiểm định Wilcoxon Signed Rank}
\begin{itemize}
    \item \textbf{Mục tiêu:}  
    So sánh sự khác biệt giữa giá trị \textit{order\_price} (giá trị gốc) và \textit{order\_total} (giá trị thực tế sau khi áp dụng giảm giá và phí giao hàng).

    \item \textbf{Phù hợp vì:}  
    \begin{itemize}
        \item Là phương pháp dành riêng cho các cặp giá trị phụ thuộc.
        \item Không yêu cầu giả định phân phối chuẩn.
    \end{itemize}
    \item \textbf{Kết quả:}  
    $p\text{-value} = 0.042 < 0.05$, khẳng định có sự khác biệt có ý nghĩa thống kê giữa \textit{order\_price} và \textit{order\_total}. Sự khác biệt này phản ánh tác động của các yếu tố như giảm giá và chi phí giao hàng.
% Điểm yếu của kiểm định Wilconixin Signed Rank
\textbf{Điểm yếu:}
\begin{itemize}
    \item Chỉ áp dụng với dữ liệu có cặp (paired data), không thể áp dụng với dữ liệu độc lập.
    \item Độ chính xác của kiểm định có thể bị ảnh hưởng nếu có quá nhiều giá trị bằng nhau trong dữ liệu.
\end{itemize}
\end{itemize}

\subsection{Mối quan hệ giữa mục tiêu và phương pháp được chọn}

\begin{itemize}
    \item Phân tích hồi quy thường dùng để xây dựng mô hình dự đoán hoặc mô hình hóa mối quan hệ tuyến tính giữa biến phụ thuộc và biến độc lập.
    \item Trong khi đó, bài toán của nhóm tập trung vào việc kiểm tra:
    \begin{itemize}
        \item \textbf{Sự khác biệt giữa các nhóm (mùa):} Dùng Kruskal-Wallis và Wilcoxon-Mann-Whitney.
        \item \textbf{Sự khác biệt giữa các giá trị trong cùng một nhóm:} Dùng Wilcoxon Signed Rank.
    \end{itemize}
    \item Việc lựa chọn kiểm định không tham số phản ánh mục tiêu phân tích cụ thể mà không phụ thuộc vào các giả định phức tạp về phân phối dữ liệu.
\end{itemize}

\noindent
\textbf{Kết luận:}  
Phân tích hồi quy không được chọn vì:
\begin{itemize}
    \item Dữ liệu vi phạm các giả định quan trọng (phân phối chuẩn, phương sai đồng nhất).
    \item Mục tiêu của nhóm không phải dự đoán giá trị mà là kiểm tra sự khác biệt thống kê.
\end{itemize}
Các kiểm định không tham số được chọn phù hợp hơn, chính xác hơn, và cung cấp kết quả giải thích trực tiếp liên quan đến câu hỏi bài toán.

\subsection{Mở rộng: Dunn Test}
% Dàn ý:
% Mở rộng phương pháp Krustal-Wallis
% Nêu ra những điểm mạnh , điếm yếu của 3 phương pháp trên
Sau khi đưa ra những điểm mạnh hay giới hạn những phương pháp nhóm đã sử dụng ở trên. Nhóm quyết định sẽ mở rộng phương pháp \textbf{Krustal-Wallis} 
Như đề cập ở trên, phương pháp kiểm định KW chỉ đưa ra được "bề nổi" hay chỉ kiểm tra sự khác biệt tổng thể giữa các nhóm.

\noindent Nhóm sẽ áp dụng phương pháp hậu nghiệm (pos-hoc) để tìm ra cụ thể nhóm nào có sự khác biệt. Nhóm đề xuất sử dụng \textbf{Dunn Test}.

\begin{lstlisting}[language=R,caption=Thực hiện phân tích Pos Hoc trong R]
dunn_result <- dunn.test(merged_data$order_price, merged_data$season, method = "bonferroni", list = TRUE)
# Tạo bảng kết quả
table <- cbind.data.frame(
  Comparison = dunn_result$comparisons,  # Các cặp so sánh (cụ thể mùa)
  Z_value = dunn_result$Z,               # Giá trị Z
  P_adjusted = dunn_result$P.adjusted    # P-value đã điều chỉnh
)
# Sắp xếp bảng theo p-value đã điều chỉnh
table <- table[order(table$P_adjusted), ]
# Tạo bảng với gt và thêm tiêu đề
table %>%
  gt() %>%
  tab_header(
    title = md("#### Kết quả phân tích pos-hoc (Dunn's Test)"),
    subtitle = "So sánh sự khác nhau giữa các nhóm mùa bằng Bonerroni"
  )

\end{lstlisting}
\subsection{Kết quả}
% Table

\begin{table}[H]
\centering
\begin{tabular}{|l|r|r|} % Cột: l (trái), r (phải), r (phải)
\hline
\textbf{Các mùa} & \textbf{Z-value} & \textbf{P.adjusted} \\ \hline
autumn - winter     & 1.4580097        & 0.4345136           \\ \hline
autumn - spring     & 0.8823710        & 1.0000000           \\ \hline
autumn - summer     & 0.6008488        & 1.0000000           \\ \hline
spring - summer     & -0.2722941       & 1.0000000           \\ \hline
spring - winter     & 0.6104089        & 1.0000000           \\ \hline
summer - winter     & 0.8656661        & 1.0000000           \\ \hline
\end{tabular}
\caption{Post-hoc Analysis Results (Dunn's Test)}
\label{tab:dunn_results}
\end{table}
% End table

    \textbf{Kết luận:}
    Sau khi xuất ra kết quả Dunn Test, tổng quan các mùa đều có khác biệt gì lớn.
\section{Nguồn Dữ Liệu}
\begin{itemize}
% \href{http://www.overleaf.com}{Something Linky} 
    \item Dữ liệu mẫu: \href{https://www.kaggle.com/datasets/muhammadshahrayar/transactional-retail-dataset-of-electronics-store}{https://www.kaggle.com/datasets/muhammadshahrayar/transactional-retail-dataset-of-electronics-store}
    \item Link R của bài tập lớn: 
    \href{https://github.com/bnhtho/btl_xtsk_241/blob/main/2333017_Assigment.R}{https://github.com/bnhtho/btl\textunderscore{xtsk}\textunderscore{241}/blob/main/2333017\textunderscore{Assigment}.R}
    
\end{itemize}
\section{Tài liệu Tham Khảo}
% Tham khảo
\begin{thebibliography}{99}
\bibitem{anderson} 
Anderson, D. R., Sweeney, D. J., \& Williams, T. A. (2016). 
\textit{Statistics for Business and Economics} (11th ed.). Cengage Learning Vietnam Company Limited.

\bibitem{peterdaagard}
Peter Dalgaard.
\textit{Introductory Statistics with R} (8th ed)
\bibitem{nguyendinhhuy}
Nguyễn Đình Huy.
\textit{Giáo trình xác suất và thống kê(2023)}, Nxb. Đại học Quốc Gia, Thành phố Hồ Chí Minh.

\end{thebibliography}

